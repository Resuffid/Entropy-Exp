\documentclass[12pt,a4paper,final,oneside,onecolumn,titlepage]{article}
\usepackage{times}
\usepackage{geometry}
\usepackage{fancyhdr}
\usepackage{setspace}
\usepackage{natbib}
\usepackage{amssymb}
\usepackage{graphicx}
\usepackage{sectsty}
\usepackage[table,xcdraw]{xcolor}
\usepackage{polski}
\usepackage[utf8]{inputenc}
\usepackage[T1]{fontenc}
\usepackage[pagewise]{lineno}
\newgeometry{tmargin=2.5cm, bmargin=2.5cm, lmargin=2.5cm, rmargin=2.5cm}
\setlength{\parindent}{3in}
\setlength{\parskip}{0pt}
\doublespacing
\sectionfont{\centering}
\renewcommand{\bibsection}{\section*{\large{\textbf{\textsc{\centering{Literatura}}}}}}

\begin{document}
\pagestyle{fancy}
\fancyhead{}
\fancyfoot{}
\chead{Nasze otoczenie i funkcje poznawcze}
\rhead{\thepage}
\bibliographystyle{apalike}
\begin{titlepage}
  \thispagestyle{empty}
  \rhead{\thepage}
  \begin{center}
  \vspace*{1cm}
  \Large
  \textbf{\textsc{Nasze otoczenie i funkcje poznawcze:\\ Badanie zależności między entropią informacyjną \textit{(H)} w otoczeniu oraz wykonaniu treningu uważności, a selektywną uwagą wzrokową.\\}}
  \vspace{1.5cm}
  \textit{Laura Plichta, Wiktor Warchałowski, Zofia Załęska\\}
  Wydział Nauk o Zdrowiu, Gdański Uniwersytet Medyczny\\
  \vspace{3cm}
  Praca zaliczeniowa z przedmiotu \\ Metodologia Badań Psychologicznych 2 \\ napisana pod kierunkiem dr Krzysztofa Basińskiego\\
  \vspace{3cm}
  Gdańsk, 20 Stycznia 2023
  \end{center}
\end{titlepage}
\begin{center}
  \vspace*{0.5cm}
  \large{\textbf{\textsc{Abstrakt}}}
\end{center}
\paragraph{}
Celem niniejszego artykułu jest zbadanie problematyki związanej z wpływem środowiska w jakim się znajdujemy na zdolności poznawcze człowieka. Artykuł ten sprawdza czy istnieje wpływ entropii informacji zaindukowanej przez różnorodność obiektów w otoczeniu i wykonaniem treningu uważności jakim jest kolorowanie mandali na selektywną uwagę wzrokową. Badanie zostało przeprowadzone na 30 osobach, które dobrowolnie zgodziły się na wzięcie udziału w eksperymencie. Indukowana entropia otoczenia
\\
\textit{Słowa kluczowe: środowisko, entropia informacji, uwaga wzrokowa, trening uważności}
\newpage
\begin{center}
\section*{\large{\textbf{\textsc{Wstęp}}}}
\end{center}
\paragraph{}
Ze względu na rozwój techniki, kończące się zasoby naturalne, zwiększająca się liczba ludności oraz inne problemy dynamicznie rozwijającego się świata wzrosło zainteresowanie badaniami zależności między człowiekiem, a jego środowiskiem. Dziedziną zajmującą się relacją ludzi i ich zachowań z różnymi modalnościami ich otoczenia oraz jego optymalizacją \citep{banka_psychologia_2018, gifford_environmental_2011}. Dostrzeganie interakcji człowieka ze środowiskiem mogą być czymś ważnym w rozwoju architektury i planowania przestrzennego, aby era antropocenu nie była stworzona destruktywnym wpływem człowieka na naturę, ale okresem w którym działamy na wspólną korzyć \citep{zalasiewicz_new_2010}. Oprócz celu zrównoważnonego rozwoju w celu odpowiedzi na zmiany klimatyczne, badacze zajmują się optymalizacją naszego najbliższego otoczenia. Przykładem takiego działania są badania \citet{lohr_interior_1996} pokazujące zależność struktury miejsca pracy z produktywnością. Jednakże analiz tego typu jest relatywnie mało, ale ich ilość wzrasta w XXI wieku \citep{spano_human_2020}. Inspirując się takim typem eksperymentów, celem niniejszego badania było sprawdzenie, jak modyfikacja środowiska pracy wpłynie na efektywność procesów poznawczych człowieka z naciskiem na selektywną uwagę wzrokową. Aby zuniwersalizować manipulację wyglądem otoczenia, zostało zastosowane pojęcie entropii informacji zgodnie z teorią Shannona. Oznacza to, że fizyczna miara nieuporządkowania i chaosu, jest interpretowana jako suma średnich prawdopodobieństw wystąpienia danego typu informacji i zdarzeń. Jest wyrażana wzorem:
\begin{center}
\begin{math}
H_f=-\displaystyle\sum_{i=1}^{n}p_i\log_2({p_i})
\end{math}
\end{center}
Gdzie $n$ oznacza ilość obiektów, $i$ to dany obiekt, zaś $p$ jest prawdopodobieństwem jego wystąpienia. Wynik powyższego działania podawany jest w bitach \citep{stamps_entropy_2004}. Dodatkowo na potrzeby analizy otoczenia entropia informacji jest liczona jako zróżnicowanie elementów architektonicznych lub dekoracyjnych \citep{stamps_entropy_2004, stamps_entropy_2002}. Ze względu na fakt badania odbioru otoczenia pod względem artystyczno-wizualnym, zdecydowano o dołączeniu kolorowania kolorowanki uważności. Jest to jeden z rodzajów treningu \textit{mindfulness} (z ang. uważność), który polega na pełnej koncentracji na przebiegu kolorowania \citep{zejmo_praktyka_2022}. W ostatnich czasach rośnie również ich popularnosć i rośnie ich rola w życiu codziennym wielu ludzi \citep{dresler_doing_2019}. Stąd dodatkowo ten artykuł sprawdza ich użyteczność co również może pomóc w rozwoju świadomości na temat uwagi i uwarunkowań jej działania. Z tego względu postawionym pytaniem badawczym jest czy zwiększona entropia informacji w kolorach elementów otoczenia oraz wykonanie treningu uważności ma wpływ na selektywną uwagą wzrokową? Przewiduje się, że Zwiększona entropia informacji w kolorach i kształtach elementów otoczenia zmniejsza czas wykonania testu selektywnej uwagi wzrokowej. Dzieje się tak ze względu na fakt, że zwiększona entropia wpływa bezpośrednio na zwiększenie przyjemności i pozytywnego pobudzenia \citep{stamps_entropy_2004, stamps_entropy_2002}, które wpływa na możliwości percepcyjne jednostki, dzięki czemu osoby, którym indukowano przyjemność i pozytywne pobudzenie lepiej radziły sobie z wykonaniem testów percepcyjnych, uwagi wzrokowej oraz ogólnie procesów poznawczych \citep{mcconnell_upbeat_2011, gavazzi_pleasure_2021}. Wykonanie testu uważności również zmniejsza czas wykonania testu selektywnej uwagi wzrokowej ze względu na fakt, że czynność kolorowania mandali pomaga w uspokojeniu się oraz poprawia uważność oraz ogólny stan osoby która koloruje (\textit{mindfulness and wellbeing}) \citep{carsley_effectiveness_2018, campenni_effects_2020}. Wysoki poziom uważności (\textit{mindfulness}) zaś podwyższa poziom uwagi wzrokowej i pomaga w skupieniu się \citep{campillo_effects_2018, sumantry_meditation_2021}. Jednakże w przypadku krótkiego zabiegu - wykonania pojedynczej kolorowanki, może nie mieć żadnego wpływu na poziom uwagi wzrokowej badanych \citep{thompson_influence_2021}.
\begin{center}
\section*{\large{\textbf{\textsc{Metoda}}}}
\end{center}
\subsection*{\normalsize{\textbf{Operacjonalizacja}}}
\paragraph{}
Zmienna niezależna, jaką jest indukowana entropia otoczenia, została zoperacjonalizowana poprzez wprowadzenie w pierwszym warunku sprecyzowanej ilości obiektów w tym samym kolorze ($\Delta H_{f1}=0$, zaś w drugim warunku tej samej ilości obiektów, lecz w różnych kolorach ($\Delta H_{f1}<\Delta H_{f2}$). Dzięki użyciu takiej samej ilości obiektów w obu warunkach entropia maksymalna ($\Delta H_{max}$) będzie równa. Manipulacja zachodzi tylko kolorem, ponieważ tylko ich entropia dodatnio koreluje z przyjemnością (pleasure) \citep{stamps_entropy_2004, stamps_entropy_2002}.\\
Druga zmienna niezależna, jaką jest trening uważności, została zoperacjonalizowana poprzez wykonanie jednego wzoru kolorowanki uważności w takim samym czasie lub kolorowaniu pustej kartki.\\
Zmienną zależną, selektywną uwagę wzrokową, zmierzono za pomocą czasu reakcji podczas wykonywania testu Eriksena (tzw. \textit{flaker test}) oraz stopnia jego poprawności.
\subsection*{\normalsize{\textbf{Osoby badane}}}
\paragraph{}
Uczestnicy zostali dobrani za pomocą doboru kuli śniegowej z populacji, zaś do losowego przypisania ich do grup niezależnych, czyli dwóch grup z różną indukowaną entropią otoczenia, wykorzystano randomizację w blokach.
\subsection*{\normalsize{\textbf{Procedura}}}
\paragraph{}
Każdy z badanych na początku został poinformowany o celu badania, jego przebiegu, dobrowolności i użycia wyników zebranych w badaniu oraz ustnie wyraził świadomą zgodę.
\paragraph{}
Za każdym razem w sali znajdowało się 30 kulek z basenu dla dzieci o różnych kolorach.
\subsubsection*{\small{\textit{Grupa z niską entropią otoczenia:}}}
\paragraph{}
Badani zostali zaproszeni pojedynczo do sali w której panował relatywny początek, a 30 kulek rozproszonych w pomieszczeniu było takiego samego koloru. Następnie badany został poproszony o kolorowanie pustej kartki przez 10 minut (warunek “A”). Po tym czasie został przeprowadzony flanker test i zmierzono czas reakcji badanego. Po przeprowadzeniu testu, badany został poproszony o kolorowanie mandali przez 10 minut (warunek “B”). Po tej aktywności badany został ponownie poproszony o wykonanie flanker testu. Taka procedura została przeprowadzona dla każdego z badanych dobranych do grupy z niską entropią jednakże u połowy badanych zmieniono kolejność manipulacji eksperymentalnej tj. najpierw przeprowadzono warunek “B”, a jako drugi warunek “A”. Do obu opcji badani zostali przydzieleni losowo przez randomizację.
\subsubsection*{\small{\textit{Grupa z wysoką entropią otoczenia:}}}
\paragraph{}
Badani zostali zaproszeni pojedynczo do sali w której panował relatywny początek, a  30 kulek rozproszonych w pomieszczeniu było trzech różnych kolorów. Następnie procedura została przeprowadzona w ten sam sposób co w grupie z niską entropią otoczenia - badani zostali poproszeni o obydwu warunków eksperymentalnych (“A” i “B”), połowa w kolejności “A”-”B”, zaś druga “B”-”A”. 
\subsection*{\normalsize{\textbf{Etyka}}}
\paragraph{}
Zgoda etyczna na przeprowadzenie badania została otrzymana od prowadzącego przedmiot Metodologia Badań Psychologicznych, realizowanym na Gdańskim Uniwersytecie Medycznym. Badanie, ani kwestionariusz demograficzny nie zbiera danych wrażliwych. Uczestnicy zostali poinformowani o celu badania oraz o jego naturze. Zapewniono ich również, iż udział w badaniu jest zupełnie dobrowolny i w każdej chwili mogą z niego zrezygnować, przez cały czas pozostając anonimowym. Pozyskane zostało też potwierdzenie, że wszyscy uczestnicy ukończyli 18 lat i mają prawo do wyrażenia samodzielnej, świadomej zgody na udział w badaniu. 
\subsection*{\normalsize{\textbf{Analiza statystyczna}}}
\paragraph{}
W celu udzielenia odpowiedzi na postawione pytanie badawcze oraz przetestowania postawionej hipotezy przeprowadzono analizy statystyczne przy użyciu języka programowania i środowiska obliczeniowego R Project for Statistical Computing \citep{r_core_team_r_2022}. Pierwszym wykonanym zabiegiem statystycznym było sprawdzenie normalności rozkładu testem Shapiro-Wilka, aby być w stanie dobrać odpowiednio następne testy statystyczne. Za poziom istotności przyjęto $\alpha$ = 0.05. 

\newpage
\bibliography{Biblio}
\end{document}