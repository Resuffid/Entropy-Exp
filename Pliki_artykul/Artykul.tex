\documentclass[12pt,a4paper,final,oneside,onecolumn,titlepage]{article}
\usepackage{times}
\usepackage{geometry}
\usepackage{fancyhdr}
\usepackage{setspace}
\usepackage{natbib}
\usepackage{float}
\usepackage{multirow}
\usepackage{amssymb}
\usepackage{graphicx}
\usepackage{sectsty}
\usepackage[table,xcdraw]{xcolor}
\usepackage{polski}
\usepackage[utf8]{inputenc}
\usepackage[T1]{fontenc}
\usepackage[pagewise]{lineno}
\newgeometry{tmargin=2.5cm, bmargin=2.5cm, lmargin=2.5cm, rmargin=2.5cm}
\setlength{\parindent}{3in}
\setlength{\parskip}{0pt}
\linenumbers
\doublespacing
\sectionfont{\centering}
\renewcommand{\bibsection}{\section*{\large{\textbf{\textsc{\centering{Literatura}}}}}}

\begin{document}
\pagestyle{fancy}
\fancyhead{}
\fancyfoot{}
\chead{Nasze otoczenie i funkcje poznawcze}
\rhead{\thepage}
\bibliographystyle{apalike}
\begin{titlepage}
  \thispagestyle{empty}
  \rhead{\thepage}
  \begin{center}
  \vspace*{1cm}
  \Large
  \textbf{\textsc{Nasze otoczenie i funkcje poznawcze:\\ Badanie zależności między entropią informacyjną \textit{(H)} w otoczeniu oraz wykonaniem treningu uważności, a selektywną uwagą wzrokową.\\}}
  \vspace{1.5cm}
  \textit{Laura Plichta, Wiktor Warchałowski, Zofia Załęska\\}
  Wydział Nauk o Zdrowiu, Gdański Uniwersytet Medyczny\\
  \vspace{3cm}
  Praca zaliczeniowa z przedmiotu \\ Metodologia Badań Psychologicznych 2 \\ napisana pod kierunkiem dr. Krzysztofa Basińskiego\\
  \vspace{3cm}
  Gdańsk, 23 Stycznia 2023
  \end{center}
\end{titlepage}
\begin{center}
  \vspace*{0.5cm}
  \large{\textbf{\textsc{Abstrakt}}}
\end{center}
\paragraph{}
Celem niniejszego artykułu jest zbadanie problematyki związanej z wpływem środowiska w jakim się znajdujemy na zdolności poznawcze człowieka. Artykuł ten sprawdza czy istnieje wpływ entropii informacji zaindukowanej przez różnorodność obiektów w otoczeniu i wykonaniem treningu uważności, jakim jest kolorowanie mandali, na selektywną uwagę wzrokową. Badanie zostało przeprowadzone na 30 osobach, które dobrowolnie zgodziły się na wzięcie udziału w eksperymencie. Indukowana entropia otoczenia została opisana jako różnorodność kulek do basenu dziecięcego w pomieszczeniu, zaś trening uważności jako wykonanie kolorowanki z mandalą. Zmienną niezależną była uwaga wzrokowa, zbadana za pomocą testu Eriksena. Średnie porównywanych grup, test t studenta oraz mieszane modele liniowe z nałożonymi kontrastami średnich marginalnych wykazały istnotną statystycznie interakcję porównywanych grup. Istotność statystyczna efektów treningu uważności oraz indukowanej entropii nie została wykazana Współczynnik d Cohena wskazał siłę efektu, którą można uznać za nieistotną w populacji.
\\
\textit{Słowa kluczowe: środowisko, entropia informacji, uwaga wzrokowa, trening uważności}
\newpage
\begin{center}
\section*{\large{\textbf{\textsc{Wstęp}}}}
\end{center}
\paragraph{}
Ze względu na rozwój techniki, kończące się zasoby naturalne, zwiększającą się liczbą ludności oraz inne problemy dynamicznie rozwijającego się świata wzrosło zainteresowanie badaniami zależności między człowiekiem, a jego środowiskiem. Dziedziną zajmującą się relacją ludzi i ich zachowań z różnymi modalnościami ich otoczenia oraz jego optymalizacją jest psychologia środowiska \citep{banka_psychologia_2018, gifford_environmental_2011}. Dostrzeganie interakcji człowieka ze środowiskiem może być czymś ważnym w rozwoju architektury i planowania przestrzennego, aby era antropocenu nie była stworzona destruktywnym wpływem człowieka na naturę, ale okresem w którym działamy na wspólną korzyć \citep{zalasiewicz_new_2010}. Oprócz celu zrównoważnonego rozwoju, aby odpowiedzieć na zmiany klimatyczne, badacze zajmują się optymalizacją naszego najbliższego otoczenia. Przykładem takiego działania są badania \citet{lohr_interior_1996} pokazujące zależność struktury miejsca pracy z produktywnością. Jednakże analiz tego typu jest relatywnie mało, ale ich ilość wzrasta w XXI wieku \citep{spano_human_2020}. Inspirując się takim typem eksperymentów, celem niniejszego badania było sprawdzenie, jak modyfikacja środowiska pracy wpłynie na efektywność procesów poznawczych człowieka z naciskiem na selektywną uwagę wzrokową. Aby zuniwersalizować manipulację wyglądem otoczenia, zostało zastosowane pojęcie entropii informacji zgodnie z teorią Shannona. Oznacza to, że fizyczna miara nieuporządkowania i chaosu, jest interpretowana jako suma średnich prawdopodobieństw wystąpienia danego typu informacji i zdarzeń. Jest wyrażana wzorem:
\begin{center}
\begin{math}
H_f=-\displaystyle\sum_{i=1}^{n}p_i\log_2({p_i})
\end{math}
\end{center}
Gdzie $n$ oznacza ilość obiektów, $i$ to dany obiekt, zaś $p$ jest prawdopodobieństwem jego wystąpienia. Wynik powyższego działania podawany jest w bitach \citep{stamps_entropy_2004}. Dodatkowo na potrzeby analizy otoczenia entropia informacji jest liczona jako zróżnicowanie elementów architektonicznych lub dekoracyjnych \citep{stamps_entropy_2004, stamps_entropy_2002}.
\paragraph{}
Ze względu na fakt badania odbioru otoczenia pod względem artystyczno-wizualnym, zdecydowano o dołączeniu kolorowania kolorowanki uważności. Jest to jeden z rodzajów treningu \textit{mindfulness} (z ang. uważność), który polega na pełnej koncentracji na przebiegu kolorowania \citep{zejmo_praktyka_2022}. Liczne badania wykazały, że uważność może być stanem przejściowym, wywołanym podczas krótkotrwałej praktyki (stan uważności) albo cechą osobowości obecną w codziennym życiu (uważność jako cecha) \citep{kiken_state_2015}. Mimo tego, że istnieje znaczna ilość badań sugerujących, że uważność poprawia samopoczucie, niewiele wskazuje na to, jak przebiega ten proces \citep{holzel_how_2011}. Jednym z argumentów jest to, że trening uważności zmienia aktywność mózgu \citep{gotink_8-week_2016} oraz przetwarzanie poznawcze \citep{zeidan_mindfulness_2010}. Trening uważności w postaci kolorowania mandali pomaga w uspokojeniu się i poprawia ogólny stan osób kolorujących \citep{carsley_effectiveness_2018, campenni_effects_2020}, którym również po wykonaniu tej czynności łatwiej jest się skupić oraz wykonywać powierzone zadania. Co więcej, mandale, a szczególnie ich środkowe punkty, są wykorzystywane do medytacji w celu zwiększenia poziomu uwagi i skupienia \citep{shankar_effectiveness_2020}. W ostatnich czasach rośnie również ich popularnosć i rośnie ich rola w życiu codziennym wielu ludzi \citep{dresler_doing_2019}. Stąd dodatkowo ten artykuł sprawdza ich użyteczność co również może pomóc w rozwoju świadomości na temat uwagi i uwarunkowań jej działania. Z tego względu postawionym pytaniem badawczym jest, czy zwiększona entropia informacji w kolorach elementów otoczenia oraz wykonanie treningu uważności ma wpływ na selektywną uwagą wzrokową? Przewiduje się, że zwiększona entropia informacji w kolorach elementów otoczenia zmniejsza czas wykonania testu selektywnej uwagi wzrokowej. Dzieje się tak ze względu na fakt, że zwiększona entropia wpływa bezpośrednio na zwiększenie przyjemności i pozytywnego pobudzenia \citep{stamps_entropy_2004, stamps_entropy_2002}, które wpływa na możliwości percepcyjne jednostki, dzięki czemu osoby, którym indukowano przyjemność i pozytywne pobudzenie lepiej radziły sobie z wykonaniem testów percepcyjnych, uwagi wzrokowej oraz ogólnie procesów poznawczych \citep{mcconnell_upbeat_2011, gavazzi_pleasure_2021}. Wykonanie testu uważności również zmniejsza czas wykonania testu selektywnej uwagi wzrokowej ze względu na fakt, że czynność kolorowania mandali pomaga w uspokojeniu się oraz poprawia uważność oraz ogólny stan osoby która koloruje (\textit{mindfulness and wellbeing}) \citep{carsley_effectiveness_2018, campenni_effects_2020}. Wysoki poziom uważności (\textit{mindfulness}) zaś podwyższa poziom uwagi wzrokowej i pomaga w skupieniu się \citep{campillo_effects_2018, sumantry_meditation_2021}. Jednakże w przypadku krótkiego zabiegu - wykonania pojedynczej kolorowanki, może nie mieć żadnego wpływu na poziom uwagi wzrokowej badanych \citep{thompson_influence_2021}.
\begin{center}
\section*{\large{\textbf{\textsc{Metoda}}}}
\end{center}
\subsection*{\normalsize{\textbf{Operacjonalizacja}}}
\paragraph{}
Zmienna niezależna, jaką jest indukowana entropia otoczenia, została zoperacjonalizowana poprzez wprowadzenie w pierwszym warunku sprecyzowanej ilości obiektów - 30 kulek z basenu dla dzieci w tym samym kolorze ($\Delta H_{f1}=0$), zaś w drugim warunku tej samej ilości kulek, lecz w 3 różnych kolorach ($\Delta H_{f1}<\Delta H_{f2}$). Dzięki użyciu takiej samej ilości obiektów w obu warunkach entropia maksymalna ($\Delta H_{max}$) była równa. Manipulacja zachodziła wyłącznie kolorem, ponieważ tylko entropia tej cechy dodatnio koreluje z przyjemnością (\textit{pleasure}) \citep{stamps_entropy_2004, stamps_entropy_2002}.\\
Druga zmienna niezależna, jaką jest trening uważności, została zoperacjonalizowana poprzez wykonanie jednego wzoru kolorowanki uważności lub kolorowanie pustej kartki przez 5 minut.\\
Zmienną zależną, selektywną uwagę wzrokową, zmierzono za pomocą czasu reakcji podczas wykonywania testu Eriksena (tzw. \textit{flaker test}) oraz stopnia jego poprawności \citep{sanders_eriksen_2002}. Został on zaprogramowany w oprogramowaniu PsychoPy \citep{peirce_psychopy2_2019}. Polegał on na jak najszybszym rozpoznaniu kierunku środkowej strzałki w układzie pięciu strzałek. Miał on 6 możliwych wariantów. W dwóch z nich wszystkie stzałki były w jednym kierunku, w kolejnych dwóch przypadkach środkowe strzałki były w innym kierunku niż pozostałe, zaś w ostatnich dwóch środkowe strzałki były na tle bodźca neurtralnego - poziomych kreskach.
\subsection*{\normalsize{\textbf{Osoby badane}}}
\paragraph{}
Uczestnicy zostali dobrani za pomocą doboru kuli śniegowej z populacji, zaś do losowego przypisania ich do grup niezależnych, czyli dwóch grup z różną indukowaną entropią otoczenia, wykorzystano randomizację w blokach. Łącznie było 30 uczestników o średniej wieku 24 lata. Większość (20 osób) stanowiły kobiety.
\subsection*{\normalsize{\textbf{Procedura}}}
\paragraph{}
Każdy z badanych na początku został poinformowany o celu badania, jego przebiegu, dobrowolności i użycia wyników zebranych w badaniu oraz ustnie wyraził świadomą zgodę.
\paragraph{}
Za każdym razem w sali znajdowało się 30 kulek z basenu dla dzieci o różnych kolorach.
\paragraph{}
\textbf{Grupa z niską entropią otoczenia.} Badani zostali zaproszeni pojedynczo do sali w której panował relatywny początek, a 30 kulek rozproszonych w pomieszczeniu było takiego samego koloru. Następnie badany został poproszony o kolorowanie pustej kartki przez 5 minut (warunek “A”). Po tym czasie został przeprowadzony flanker test i zmierzono czas reakcji badanego. Po przeprowadzeniu testu, badany został poproszony o kolorowanie mandali przez 5 minut (warunek “B”). Po tej aktywności badany został ponownie poproszony o wykonanie flanker testu. Taka procedura została przeprowadzona dla każdego z badanych dobranych do grupy z niską entropią jednakże u losowej części badanych zmieniono kolejność manipulacji eksperymentalnej tj. najpierw przeprowadzono warunek “B”, a jako drugi warunek “A”. Do obu opcji badani zostali przydzieleni losowo przez randomizację.
\paragraph{}
\textbf{Grupa z wysoką entropią otoczenia.} Badani zostali zaproszeni pojedynczo do sali w której panował relatywny początek, a  30 kulek rozproszonych w pomieszczeniu było trzech różnych kolorów. Następnie procedura została przeprowadzona w ten sam sposób co w grupie z niską entropią otoczenia - badani zostali poproszeni o obydwu warunków eksperymentalnych (“A” i “B”), część w kolejności “A”-”B”, zaś reszta -  “B”-”A”.
\subsection*{\normalsize{\textbf{Etyka}}}
\paragraph{}
Zgoda etyczna na przeprowadzenie badania została otrzymana od prowadzącego przedmiot Metodologia Badań Psychologicznych, realizowanym na Gdańskim Uniwersytecie Medycznym. Ani badanie, ani kwestionariusz demograficzny nie zbiera danych wrażliwych. Uczestnicy zostali poinformowani o celu badania oraz o jego naturze. Zapewniono ich również, iż udział w badaniu jest zupełnie dobrowolny i w każdej chwili mogą z niego zrezygnować, przez cały czas pozostając anonimowym. Pozyskane zostało też potwierdzenie, że wszyscy uczestnicy ukończyli 18 lat i mają prawo do wyrażenia samodzielnej, świadomej zgody na udział w badaniu. Dodatkowo nie zbierano i nie wykorzystano kolorowanek przez badanych.
\subsection*{\normalsize{\textbf{Analiza statystyczna}}}
\paragraph{}
W celu udzielenia odpowiedzi na postawione pytanie badawcze oraz przetestowania postawionej hipotezy, przeprowadzono analizy statystyczne przy użyciu języka programowania i środowiska obliczeniowego R Project for Statistical Computing \citep{r_core_team_r_2022}. Pierwszym wykonanym zabiegiem statystycznym było sprawdzenie normalności rozkładu testem Shapiro-Wilka oraz testem Kołmogorowa-Smirnowa, aby być w stanie dobrać odpowiednio następne testy statystyczne. Za poziom istotności przyjęto $\alpha = 0.05$. Wartość \textit{p} obydwu testów wyniosła $p < 2.2*10^{-16}$ co oznacza, że została przyjęta hipoteza zerowa mówiąca o rozkładzie odbiegającym od rozkładu normalnego. Graficznym przedstawieniem rozkładu uzyskanych wyników jest poniższy histogram \textit{(Rysunek 1)}. Jednakże w trakcie wstępnej analizy danych zauważono, że wszyskie wartości czasiu reakcji wynoszące powyżej 1 sekundy są wartościami odstającymi - znajdują się poza $3SD$ oraz $1.5IQR$. Dodatkowo ilość obserwacji jest bardzo duża, a z tego powodu test normalności rozkładu jest niewymierny i niekonieczny do policzenia co wynika z centralnego twierdzenia granicznego \citep{kwak_central_2017}. Rozkład wyników bez wartości odstających przedstawiony jest na \textit{Rysunku 2}. Z tego powodu, w celu statystycznego opisania otrzymanych wyników posłużono się testami parametrycznymi oraz analizy modelu liniowego. Dodatkowo przeprowadzono pomiar siły efektu za pomocą współczynnika d Cohena. Mówi on o "stopniu do jakiego badane zjawisko istnieje" \citep[s. 5]{cohen_statistical_1977}. 
\begin{figure}[H]
\centering
\caption{Histogram przedstawiający rozkład wyników z wartościami odstającymi}
\includegraphics[scale=0.5]{hist1}
\label{Rysunek}
\end{figure}
\begin{figure}[H]
\centering
\caption{Histogram przedstawiający rozkład wyników bez wartości odstających}
\includegraphics[scale=0.5]{hist2}
\label{Rysunek}
\end{figure}
\begin{center}
\section*{\large{\textbf{\textsc{Wyniki}}}}
\end{center}
\paragraph{}
Po sprawdzeniu normalności rozkładu i wybraniu typu testów statystycznych, zostały policzone podstawowe statystyki opisowe dla zbioru danych nieodbiegających od rozkładu normalnego - miary tendencji centralnej, zmienności oraz asymetrii. Były to średnia, odchylenie standardowe, klasyczny współczynnik zmienności oraz klasyczny współczynnik asymetrii. Dla całego zbioru danych średnia ($M$) wyniosła $0.49$, odchylenie standardowe ($SD$) wyniosło $0.13$, współczynnik zmienności ($V_x$) miał wartość $25.57$, zaś asymetria ($A_{S_{x}}$) wyniosła $1.16$. Wyniki powyższych cech statystycznych dla 4 badanych grup zostały przedstawione w \textit{Tabeli 1}. Dodatkowo narysowano wykres \textit{box and whiskers plot} dla powyższych grup \textit{(Rysunek 3)}.
\begin{table}[H]
\caption{Statystyki opisowe}
\centering
\begin{tabular}{l c c c c}
\hline\hline
Grupa & $M$ & $SD$ & $V_x$ & $A_{S_{x}}$ \\ [0.5ex]
\hline
Mandala i wysoka entropia&0.50&0.13&26.68&1.18 \\
Mandala i niska entropia&0.50&0.13&25.46&0.94 \\
Kartka i wysoka entropia&0.48&0.11&23.25&1.30 \\
Kartka i niska entropia&0.49&0.13&26.32&1.16 \\ [1ex]
\hline
\end{tabular}
\label{Tabela}
\end{table}
\begin{figure}[H]
\centering
\caption{Box and whiskers plot przedstawiający rozkład wyników w grupach}
\includegraphics[scale=0.5]{box1}
\label{Rysunek}
\end{figure}
\paragraph{}
Poza obliczeniem podstawowych statystyk opisowych przeprowadzono testy statystyczne badające związek pomiędzy badanymi zmiennymi. Był to test t studenta w warunku grup zależnych i grup niezależnych. Były to testy badające osobne zależności między badanymi zmiennymi. Dodatkowo przeprowadzono analizę wariancji dla wszystkich badanych zmiennych i ich interakcji za pomocą mieszanego modelu liniowego \textit{(linear mixed model)} oraz zbadano kontrasty za pomocą średnich estymowanych \textit{(estimated marginal means)}. Wartości $p$ dla testu t studenta oraz modelu liniowego zostały przedstawione w \textit{Tabeli 2}, zaś wyniki kontrastów modelu liniowego zostały przedstawione w \textit{Tabeli 3}.
\begin{table}[H]
\caption{Wartości $p$ obliczonych testów statystycznych dla zależności}
\centering
\begin{tabular}{l c c c}
\hline\hline
 & Entropia & Kolorowanka & Interakcja \\ [0.5ex]
\hline
\multirow{4}{*}{Czas reakcji}& \multicolumn{3}{c}{Test t studenta} \\
 &$0.014^*$&$0.22$&- \\ [3ex]
 & \multicolumn{3}{c}{Mieszany model liniowy}\\
 &$0.52$&$8.55*10^{-5*}$&$0.037^*$ \\
\hline
\multicolumn{4}{l}{\footnotesize{$^{*} - p<0.05$}}
\end{tabular}
\label{Tabela}
\end{table}
\begin{table}[H]
\caption{Wartości $p$ kontrastów nałożonych na model liniowy}
\centering
\begin{tabular}{c c c}
\hline\hline
\multicolumn{2}{c}{Kontrast (Entropia, Mandala)} & Wartość $p$ \\ [0.5ex]
\hline
N nie & W nie & $0.77$ \\
N nie & N tak & $0.56$ \\
N nie & W tak & $1.00$ \\
W nie & N tak & $0.60$ \\
W nie & W tak & $1.00*10^{-4*}$ \\
N tak & W tak & $0.99$ \\
\hline
\multicolumn{3}{l}{\footnotesize{$^{*} - p<0.05$}}
\end{tabular}
\label{Tabela}
\end{table}
\paragraph{}
Powyższe wyniki pokazują, że test t studenta wykazał istotność statystyczną dla zależności entropii i czasu reackji oraz model liniowy wykazał istotność statystyczną zależności kolorowania mandali z czasem reakcji ze względu na zachodzącą istotnie statystyczną interakcję. Kontrasty wykazały, że kolorowanie mandali istotnie zwiększa czas reakcji tylko w warunku wysokiej entropii.
\paragraph{}
Wartości siły efektu dla obu grup są nieistotne. Dla wpływu kolorowania na czas reakcji w teście uwagi wartość współczynnika d Cohena wyniosła $0.037$, zaś dla wpływu entropii otoczenia na czas reakcji w teście uwagi wzrokowej wartość współczynnika d Cohena wyniosła $0.082$.
\paragraph{}
Dodatkowo, pomimo braku tej predykcji w pytaniu badawczym oraz w hipotezie, zauważono iż może istnieć związek pomiędzy typem bodźca wzrokowego, a czasem reakcji oraz, że w przypadku faktu wykonania bądź niewykonania treningu uważności wyniki w teście uwagi się różnią. W tym celu przeprowadzono test ANOVA oraz ponownie mieszany model liniowy \textit{(linear mixed model)} ze zbadaniem kontrastów ze średnich estymowanych z modelu \textit{(estimated marginal means)}. Wynika z nich, że istnieje istotnie statystyczny związek między typem bodźca, a czasem reakcji - wartość $p$ jednoczynnikowego testu ANOVA wyniosła $p<2*10{-16}$. Model liniowy z nałóżonymi kontrastami, zaś wykazał istotną interakcję między typem bodźca, a faktem kolorowania mandali ($p<0.05$). Graficzne różnice zostały przedstawione na poniższym wykresie \textit{box and whiskers plot} na \textit{Rysunku 4}. Wyniki dopasowania modelu liniowego oraz nałożonych kontrastów załączone zostały w \textit{Załączniku 1}.
\begin{figure}[H]
\centering
\caption{Box and whiskers plot przedstawiający mediany wyników w grupach}
\includegraphics[scale=0.5]{box2}
\label{Rysunek}
\end{figure}
\paragraph{}
Podsumowując wszystkie powyższe wyniki, przyjęto hipotezę alternatywną ($H_1$) mówiącą o istniejącym statystycznym związku dla interakcji entropii otoczenia z faktem wykonania kolorowanki uważności, dla interakcji typu bodźca wzrokowego z wykonaniem kolorowanki oraz dla samego efektu głównego typu bodźca wzrokowego - w warunku wysokiej entropii, wykonanie kolorowanki wiąże się z wyższymi czasami reakcji, tak samo jak w warunku nieprzystającego (ang. \textit{incongruent}) bodźca, a sam nieprzystający (ang. \textit{incongruent}) bodziec również zwiększa czasy reakcji.
\begin{center}
\section*{\large{\textbf{\textsc{Dyskusja}}}}
\end{center}
\paragraph{}
Badanie miało na celu sprawdzenie wpływu entropii informacji zaindukowanej przez różnorodność obiektów w otoczeniu i wykonaniem treningu uważności na selektywną uwagę wzrokową. Postawiono hipotezę o pozytywnym wpływie wysokiej entropii otoczenia oraz wykonaniem treningu uważności na selektywną uwagę wzrokową. Spodziewano się zatem wyników, które świadczyłyby o tym, że osoby wykonujące test uwagi Eriksena w środowisku o wysokiej entropii po wcześniejszym wykonaniu kolorowanki uważności będą miały lepsze (niższe) czasy reakcji niż osoby malujące najpierw pustą kartkę lub znajdują się w pomieszczeniu o niskiej entropii. Jednakże, uzyskane wyniki nie potwierdzają założeń postawionej w badaniu hipotezy. Może to stanowić sprzeczność z wynikami większości dotychczasowych badań szukających podobnych zależności, ale jednocześnie ciekawe ich uzupełnienie. Okazuje się, że nie istnieją efekty główne indukowanej entropii otoczenia oraz treningu uważności na selektywną uwagę wzrokową. Jednakże, badanie wykazało istotną interakcję badanych zmiennych niezależnych oraz efekt prosty kolorowania mandali w warunku wysokiej entropii otoczenia. Dodatkowo poza hipotezą został wykazany efekt główny typu bodźca wzrokowego oraz jego interakcję z treningiem uważności i jej efekt prosty w warunku bodźca nieprzystającego (ang. \textit{incongruent}).
\paragraph{}
Psychologia środowiska i optymalizacja środowiska pracy stają się coraz ważniejsze w rozwoju architektury i planowania przestrzennego. Mają one za zadanie poprawę umiejętności i efektywności własnej pracy \citep{banka_psychologia_2018}. Cały czas jako ludzie badamy nowe uwarunkowania środowiskowe, nowe sposoby optymalizacji naszego otoczenia. Szukamy interakcji jakie mogą przyczynić się do relacji między ludźmi, a ich środowiskiem \citep{spano_human_2020}. Niezależnie od tego jak zmodyfikowane jest środowisko, zmiany mogą się przyczynić do innego postrzegania otoczenia, a jednocześnie mogą w żadnym stopniu nie zmienić wewnętrznej atrybucji człowieka wobec otoczenia w jakim się znajduje. Może też nie zmienić jego dyspozycji psychicznej w danym momencie. W badaniu założono, że procesy poznawcze związane z selektywną uwagą poprawią się dzięki zaindukowaniu wysokiej entropii otoczenia. Jednakże, założenie to zostało dokonane pośrednio, ponieważ entropia wyrażona przez różnorodność kolorów może powodować wzrost pozytywnego pobudzenia \citep{stamps_entropy_2004, stamps_entropy_2002}, a dopiero zaś pozytywne pobudzenie skutkuje poprawą funkcji poznawczych \citep{gavazzi_pleasure_2021}. Dodatkowo przyjemność i pobudzenie w badaniach indukowano bodźcami słuchowymi \citep{mcconnell_upbeat_2011}, więc być może tylko ta modalność znacząco wpływa na ludzkie procesy spostrzegania. Dodatkowo, sam fakt przyzwyczajenia do danego typu otoczenia oraz indywidualne preferencje mogą wpłynąć na to jak ludzie zachowują się w danym środowisku. Ponadto przez bodźce wzrokowe jesteśmy dużo bardziej otoczeni w codziennym życiu niż specyficzne bodźce słuchowe. Porównanie tych dwóch modalności może być impulsem do kolejnych badań.
\paragraph{}
Trening uważności stał się w dzisiejszych czasach bardzo rozsławionym tworem psychologii popularnej i staje się coraz ważniejszą częścią codzienności naszej populacji \citep{dresler_doing_2019}. Badania pokazują, że wykonywanie treningu uważności pomaga w uspokojeniu się i poprawia ogólny stan osób kolorujących \citep{carsley_effectiveness_2018, campenni_effects_2020}, którym również po wykonaniu tej czynności łatwiej jest się skupić oraz wykonywać powierzone zadania. Jednakże, czas po wykonaniu tej czynności po jakim utrzymuje się stan podwyższonej uwagi i gotowości organizmu oraz na ile się on utrzymuje, nie jest znany, więc możliwym wyjaśnieniem braku efektu głównego kolorowania mandali jest to, że efekt tego pobudzenia nie był wystarczająco trwały. Dodatkowo fakt uspokojenia się po wykonaniu kolorowanki \citep{campenni_effects_2020} może wpłynąć pozytywnie na uwagę, ale może zaskutować opieszałością, co może wyjaśnić wyższe wartości czasu reakcji. Dodatkowo nie wiemy w pełni jak przebiega proces podwyższenia samopoczucia oraz uwagi przez kolorowanie mandali \citep{holzel_how_2011}. Wiemy, że proces ten zmienia aktywność mózgu \citep{gotink_8-week_2016}, jednak nie znamy pełnych obszarów i połączeń synaptycznych, które modelują te procesy. Możliwe jest to, że potrzeba jest czasu na stworzenie odpowiednich połączeń i krótka aktywność kolorowania może nie mieć efektu. Ponadto, może być to impulsem do kolejnych badań neurologicznych i neuropsychologicznych.
\paragraph{}
Analiza modelu liniowego wykazała interakcję działania entropii i kolorowanki uważności, pokazując, że kolorowanie pozytywnie wpływa na selektywną uwagę wzrokową w warunku wysokiej entropii. Może to oznaczać, że tylko połączenie tych dwóch bodźców wzmaga pobudzenie, przyjemność i dobre samopoczucie (\textit{wellbeing}) do takiego stopnia, że człowiek jest w stanie poprawić swoje procesy poznawcze przy bardzo wysokim pobudzeniu. Dodatkowo być może duża ilość obiektów artystycznych wzmaga skupienie uwagi. Innym możliwym wyjaśnieniem jest fakt, że uspokojenie się poprzez trening uważności \citep{campenni_effects_2020}, może zostać zmienione poprzez pobudzenie w wyniku przebywania w różnorodnym otoczeniu. Fakt wielu różnych źródeł pobudzenia, również może zostać zbadany w dalszych badaniach naukowych.
\paragraph{}
Pomimo braku tej predykcji w pytaniu badawczym oraz w hipotezie, wykazano istotne zależności między faktem kolorowania w przypadku reakcji na bodziec nieprzystający. Może to wynikać z obcowania ze szczegółowymi kolorowankami - obcowanie ze szczegółami, może wpływać pozytywnie na spostrzeganie szczegółów gdzieś indziej. Dodatkowo w badaniach nad mandalami sprawdzano wykonanie powierzonych zadań szkolnych \citep{carsley_effectiveness_2018} co może bardziej się pokrywać z zadaniem rozpoznania bodźca nieprzystającego. Dodatkowo w tej samej analizie wykazano, że typ nieprzystający zawsze wiąże się z dłuższymi czasami reakcji, co potwierdza teorię dotyczącą testu Eriksena \citep{sanders_eriksen_2002}.
\subsection*{\normalsize{\textbf{Ewaluacja}}}
\paragraph{}
Powyższe badanie jednakże posiada swoje ograniczenia. W przypadku krótkiej czynności - pokolorowania pojedynczej mandali - wpływ na poziom uwagi wzrokowej badanych może nie być wykrywalny \citep{thompson_influence_2021}, co może być jednym z możliwych wyjaśnień niepotwierdzenia założonej hipotezy efektu głównego treningu uważności. Dodatkowo nie są znane uwarunkowania różnic płciowych w zależności od rodzaju kolorowania, jak również obycie się i doświadczenia badanych z mandalami, a zmienne te nie zostały skontrolowane w badaniu. Takie zmienne indywidualne mogły wpłynąć na przedstawione wyniki, ale równocześnie mogą one stanowić przedmiot zainteresowania kolejnych badań. Inną słabą stroną badania może być niska trafność ekologiczna. Indukowanie entropii otoczenia za pomocą kulek z basenu dla dzieci, nie jest sytuacją typową dla człowieka i nie spotyka się z nią na co dzień.
\paragraph{}
Silną stroną badania było zachowanie hermetycznego zamkniętego środowiska, takiego samego dla każdego badanego w którym modyfikacja otoczenia była zachowana w niezmienionym stanie. Każdy badany otrzymał taką samą kolorowankę, a wszystkie parametry (czas i ekspozycja na bodźce) były kontrolowane maszynowo przez program komputerowy.
\newpage
\bibliography{Biblio}
\newpage
\section*{\large{\textbf{\textsc{Załącznik 1}}}}
\begin{table}[h]
\caption{Wyniki dopasowania modelu liniowego oraz nałożonych kontrastów}
\centering
\scalebox{0.8}{
\begin{tabular}{c c c}
\hline\hline
\multicolumn{3}{c}{\textbf{Wartości $p$ dopasowania modelu liniowego}} \\
\multicolumn{2}{l}{Zmienna} & wartość $p$ \\
\hline
\multicolumn{2}{l}{Entropia} & $0.50$ \\
\multicolumn{2}{l}{Typ 1} & $<2*10^{-16*}$ \\
\multicolumn{2}{l}{Typ 2} & $<2*10^{-16*}$ \\
\multicolumn{2}{l}{Mandala} & $<2*05^{-5*}$ \\ 
\multicolumn{2}{l}{Mandala:Typ1} & $0.40$ \\
\multicolumn{2}{l}{Mandala:Typ 2} & $0.0092^*$ \\
\multicolumn{2}{l}{Entropia:Typ 1} & $0.29$ \\
\multicolumn{2}{l}{Entropia:Typ 2} & $0.44$ \\
\hline
\multicolumn{3}{c}{\textbf{Wartości $p$ kontrastów}} \\
\multicolumn{2}{c}{Kontrast (entropia, typ)} & wartość $p$ \\
\hline
N Congurent & W Congurent & $1.00$ \\
N Congurent & N Incongurent & $0.0001^*$ \\
N Congurent & W Incongurent & $0.014^*$ \\
N Congurent & N Neutral & $0.97$ \\
N Congurent & W Neutral & $0.94$ \\
W Congurent & N Incongurent & $0.0001^*$ \\
W Congurent & W Incongurent & $0.0001^*$ \\
W Congurent & N Neutral & $1.00$ \\
W Congurent & W Neutral & $0.39$ \\
N Incongurent & W Incongurent & $0.96$ \\
N Incongurent & N Neutral & $0.0001^*$ \\
N Incongurent & W Neutral & $0.0001^*$ \\
W Incongurent & N Neutral & $0.0072^*$ \\
W Incongurent & W Neutral & $0.0001^*$ \\
N Neutral & W Neutral & $0.98$ \\
\hline
\multicolumn{2}{c}{Kontrast (mandala, typ)} & wartość $p$ \\
\hline
nie Congurent & tak Congurent & $0.47$ \\
nie Congurent & nie Incongurent & $0.0001^*$ \\
nie Congurent & tak Incongurent & $0.0001^*$ \\
nie Congurent & nie Neutral & $0.92$ \\
nie Congurent & tak Neutral & $1.00$ \\
tak Congurent & nie Incongurent & $0.0001^*$ \\
tak Congurent & tak Incongurent & $0.0001^*$ \\
tak Congurent & nie Neutral & $0.063$ \\
tak Congurent & tak Neutral & $0.49$ \\
nie Incongurent & tak Incongurent & $0.0001^*$ \\
nie Incongurent & nie Neutral & $0.0001^*$ \\
nie Incongurent & tak Neutral & $0.0001^*$ \\
nie Incongurent & nie Neutral & $0.0001^*$ \\
nie Incongurent & tak Neutral & $0.0001^*$ \\
nie Neutral & tak Neutral & $0.91$ \\
\hline
\multicolumn{3}{l}{\footnotesize{$^{*} - p<0.05$}}
\end{tabular}}
\label{Tabela}
\end{table}
\end{document}